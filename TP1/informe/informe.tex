\documentclass[a4paper]{article}

\usepackage[utf8]{inputenc}
\usepackage[spanish]{babel}

\usepackage{fancyhdr,graphicx,listings,amsmath}
\usepackage[colorlinks=true,linkcolor=black,urlcolor=blue,bookmarksopen=true]{hyperref}

\newcommand{\materia}{[75.06 / 95.58] Organización de Datos}
\newcommand{\trabajo}{Trabajo Práctico 1: Análisis Exploratorio de Datos}
\newcommand{\trabajoheader}{TP1}
\newcommand{\cuatri}{2c2018}
\newcommand{\cuatrimestre}{Segundo cuatrimestre de 2018}
\newcommand{\grupo}{Grupo Datatouille}
\newcommand{\repo}{https://github.com/FdelMazo/7506-Datos/}
\newcommand{\alumnos}{
    del Mazo, Federico & 100029 & delMazoFederico@gmail.com\\
    Bojman, Camila & 101055 &  camiboj@gmail.com\\
    Hortas, Cecilia & 100687 & ceci.hortas@gmail.com\\
    Souto, Rodrigo & 97649 & ceci.hortas@gmail.com\\
}
\newcommand{\curso}{Curso 01}
\newcommand{\bigO}[1]{\(\mathcal{O}(#1)\)}
\newcommand{\docentes}{
    \item Argerich, Luis Argerich
    \item Golmar, Natalia
    \item Martinelli, Damina Ariel
    \item Ramos Mejia, Martín Gabriel
}

\hypersetup{
    pdftitle={\trabajo},
	pdfsubject={\materia},
	pdfauthor={\grupo},
}

\pagestyle{fancy}
\fancyhf{}
\fancyhead[L]{\materia}
\fancyhead[R]{\trabajoheader - \cuatri}
\renewcommand{\headrulewidth}{0.4pt}
\fancyfoot[C]{\thepage}
\renewcommand{\footrulewidth}{0.4pt}

\begin{document}

\pagenumbering{roman}
\begin{titlepage}
	\hfill\includegraphics[width=6cm]{fiuba.jpg}
    \begin{center}
    \vfill
    \Huge \textbf{\trabajo}
    \vskip2cm
    \Large \materia\\
    \cuatrimestre
    \vfill
    \begin{flushleft} 
    \grupo
    \end{flushleft}
    \begin{tabular}{|l|c|r|}
	\hline
	Alumno & Padrón & Mail\\
	\hline \hline
    \alumnos
	\hline
	\end{tabular}
    \begin{flushleft} 
    \url{\repo}
    \end{flushleft}
    \vskip1cm
    \end{center}
    \curso
    \begin{itemize}
        \docentes
    \end{itemize}
\end{titlepage}

\tableofcontents
\newpage
\pagenumbering{arabic}
\setcounter{page}{1}

\section{Introducción}
\subsection{Ejemplo}
\subsubsection{Ejemplo2}

 Lorem ipsum dolor sit amet, consectetur adipiscing elit. Proin ultricies justo nisi, in ultrices lorem sollicitudin sed. Donec diam velit, aliquet et neque ac, tempus condimentum dolor. Maecenas scelerisque malesuada dignissim. Morbi sollicitudin est eu varius vestibulum. Duis sollicitudin non sapien quis iaculis. Quisque et luctus massa. In vitae odio vitae erat dapibus laoreet vitae in massa. Interdum et malesuada fames ac ante ipsum primis in faucibus. Nunc eu nunc tellus. Proin dignissim venenatis justo, vel rhoncus nibh commodo tincidunt. Suspendisse eleifend massa eget ligula viverra lacinia. Mauris egestas nisl a tincidunt rhoncus. 

Ejemplo de ecuación matemática: 

\[ \text{Pitágoras: }x^2 + y^2 = z^2 \]

También sobre el mismo texto: \( \text{Pitágoras: }x^2 + y^2 = z^2 \)

\subsubsection{Aun más ejemplos}

Ejemplo de items:

\begin{itemize}
\item Primer item
\item Segundo item
\end{itemize}

\section{Más ejemplos}

Ejemplo de notas al pie de página\footnote{Pie de página}.

Ejemplo de citas \cite{Cormen}.

Ejemplo de urls: \url{http://www.example.com}.

\section{Ya no se}

La figura \ref{fig:fiuba} es una gran figura.
\begin{figure}[!hb]
  \centering
    \includegraphics[width=0.5\textwidth]{fiuba.jpg}
  \caption{Caption de figura}
  \label{fig:fiuba}
\end{figure}

\newpage
\appendix

\section{Primer apéndice}

Todo el trabajo fue codificado en Python 3.6.1.

\begin{lstlisting}
def pitagoras(x,y):
    return x**2 + y**2
\end{lstlisting}

Este algoritmo es \bigO{1} pero podría haber sido \bigO{n^2} si no supiese codear.

\newpage


\phantomsection 
\addcontentsline{toc}{section}{Referencias} 
\begin{thebibliography}{9}
\bibitem{Cormen} 
Cormen, Thomas H.; Leiserson Charles E.; Rivest Ronald L; Stein Clifford:
\textit{Introduction to Algorithms}. 
The MIT Press, 2009

\end{thebibliography}
\phantomsection 
\addcontentsline{toc}{section}{Bibliografía}
\renewcommand\refname{Bibliografía}
\begin{thebibliography}{9}

\bibitem{Kruse} 
Kruse, Robert Leroy; Tondon, Clovis L.; Leung, Bruce P: 
\textit{Data structures and program design in C}
Prentice-Hall, 1997.

\end{thebibliography}
\end{document}
