\documentclass[a4paper]{article}

\usepackage[utf8]{inputenc}
\usepackage[spanish]{babel}
\usepackage{float}
\usepackage{fancyhdr,graphicx,listings,amsmath}
\usepackage[colorlinks=true,linkcolor=black,urlcolor=blue,bookmarksopen=true]{hyperref}

\newcommand{\materia}{[75.06 / 95.58] Organización de Datos}
\newcommand{\trabajo}{Trabajo Práctico 1: Análisis Exploratorio de Datos}
\newcommand{\trabajoheader}{TP1}
\newcommand{\cuatri}{2c2018}
\newcommand{\cuatrimestre}{Segundo cuatrimestre de 2018}
\newcommand{\grupo}{Grupo Datatouille}
\newcommand{\repo}{https://github.com/FdelMazo/7506-Datos/}
\newcommand{\kernel}{https://kaggle.com/datatouille2018/7506-TP1/}
\newcommand{\alumnos}{
    del Mazo, Federico & 100029 & delmazofederico@gmail.com\\
    Bojman, Camila & 101055 &  camiboj@gmail.com\\
    Hortas, Cecilia & 100687 & ceci.hortas@gmail.com\\
    Souto, Rodrigo & 97649 & rnsoutob@gmail.com\\
}
\newcommand{\curso}{Curso 01}
\newcommand{\docentes}{
    \item Argerich, Luis Argerich
    \item Golmar, Natalia
    \item Martinelli, Damina Ariel
    \item Ramos Mejia, Martín Gabriel
}

\hypersetup{
    pdftitle={\trabajo},
	pdfsubject={\materia},
	pdfauthor={\grupo},
}

\pagestyle{fancy}
\fancyhf{}
\fancyhead[L]{\materia}
\fancyhead[R]{\trabajoheader - \cuatri}
\renewcommand{\headrulewidth}{0.4pt}
\fancyfoot[C]{\thepage}
\renewcommand{\footrulewidth}{0.4pt}

\begin{document}
\pagenumbering{gobble}

\begin{titlepage}
	\hfill\includegraphics[width=6cm]{fiuba.jpg}
    \begin{center}
    \vfill
    \Huge \textbf{\trabajo}
    \vskip2cm
    \Large \materia\\
    \cuatrimestre
    \vfill
    \begin{flushleft} 
    \grupo
    \end{flushleft}
    \begin{tabular}{|l|c|r|}
	\hline
	Alumno & Padrón & Mail\\
	\hline \hline
    \alumnos
	\hline
	\end{tabular}
    \begin{flushleft} 
    \url{\repo} \\
    \url{\kernel}
    \end{flushleft}
    \vskip1cm
    \end{center}
    \curso
    \begin{itemize}
        \docentes
    \end{itemize}
\end{titlepage}
\pagenumbering{roman}
\tableofcontents
\newpage
\pagenumbering{arabic}
\setcounter{page}{1}

\section{Introducción}

Se propone analizar en el presente informe los datos obtenidos de usuarios que visitaron \texttt{www.trocafone.com}, un sitio de e-commerce de compra y venta de celulares reacondicionados, con operaciones principalmente en Brasil. Para ello, la empresa Trocafone nos proporcionó acceso a los datos a través del archivo \texttt{events.csv}. Estos datos corresponden al período de tiempo comprendido entre el 1 de enero del 2018 al 16 de junio del 2018.

El objetivo de este informe es realizar un análisis de dicho set de datos para obtener un listado de \textit{insights} aprendidos sobre los mismos. Se propone específicamente:
 
 \begin{itemize}
	 \item Descubrir features en el campo \texttt{model}
	 \item Identificar patrones de usuarios que realizan checkouts y conversiones
	 \item Analizar las búsquedas que realizan los usuarios y las \textit{keywords} utilizadas
	 \item Analizar los distintos lugares de dónde se originan las visitas a Trocafone
	 \item Descubrir features jerarquizando alguno de los campos disponibles
	 \item \textbf{agregar más items a medida que desarrollamos el informe}
\end{itemize}

Finalmente se busca realizar un aporte a la empresa Trocafone con datos que sirvan para mejorar sus servicios. 

\section{Información general sobre los datos}

En el archivo proporcionado por la empresa Trocafone se observan las siguientes columnas:

\begin{itemize}
	\item \textbf{timestamp}: Fecha y hora cuando ocurrió el evento.
	\item \textbf{event}: Tipo de evento.
	\item \textbf{person}: Identificador de cliente que realizó el evento.
	\item \textbf{url}: Url visitada por el usuario.
	\item \textbf{sku}: Identificador de producto relacionado al evento.
	\item \textbf{model}: Nombre descriptivo del producto incluyendo marca y modelo.
	\item \textbf{condition}: Condición de venta del producto.
	\item \textbf{storage}: Cantidad de almacenamiento del producto.
	\item \textbf{color}: Color del producto.	
	\item \textbf{skus}: Identificadores de productos visualizados en el evento.
	\item \textbf{search\_term}: Téŕminos de búsqueda utilizados en el evento.
	\item \textbf{static\_page}: Identificador de página estática visitada.
	\item \textbf{campagn\_source}: Origen de campaña, si el tráfico se originó de una campaña de marketing.
	\item \textbf{text\_engine}: Motor de búsqueda desde donde se originó el evento, si aplica.
	\item \textbf{channel}: Tipo de canal desde donde se originó el evento.
	\item \textbf{new\_vs\_returning}: Indicador de si el evento fue generado por un usuario nuevo (New) o por un usuario que previamente había visitado el sitio (Returning) según el motor de analytics.
	\item \textbf{city}: Ciudad desde donde se originó el evento.
	\item \textbf{region}: Región desde donde se originó el evento.
	\item \textbf{country}: País desde donde se originó el evento.
	\item \textbf{device\_type}: Tipo de dispositivo desde donde se genero el evento.
	\item \textbf{screen\_resolution}: Resolución de pantalla que se está utilizando en el dispositivo desde donde se generó el evento.
	\item \textbf{operating\_system\_version}: Versión de sistema operativo desde donde se originó el evento.
	\item \textbf{browser\_version}: Versión del browser utilizado en el evento.
\end{itemize}

\section{Datasets adicionales incoporados para el análisis}

Se utilizan adicionalmente los siguientes archivos para realizar el análisis:

\begin{itemize}
	\item {Mapas de Brazil y Estados unidos: Sacados de Geonames \footnote{\url{http://www.geonames.org/}} }
	\item \texttt{brands.csv} para extraer información del modelo
	\footnote{Detallado en \texttt{anexo.ipynb}}
	\item \texttt{os.csv} para extraer información del sistema operativo
	\footnote{Detallado en \texttt{anexo.ipynb}}
    \item \texttt{browsers.csv} para extraer información de la versión del browser
	\footnote{Detallado en \texttt{anexo.ipynb}}
\end{itemize}


\section{Ejecución}

El trabajo fue realizado en Anaconda \footnote{\url{https://anaconda.org/}}. Para poder replicar el trabajo, hay que también instalar las siguientes librerías adicionales:

\begin{itemize}
	\item{Squarify \footnote{\url{https://github.com/laserson/squarify}}: Para los treemaps.}
	\item{Geopandas \footnote{\url{http://geopandas.org/}}: Para poder graficar sobre mapas geográficos.}
	\item{Wordcloud \footnote{\url{https://github.com/amueller/word_cloud/}}: Para poder visualizar los términos más buscados.}
\end{itemize}

Estos pueden ser instalados con los siguientes comandos: 
\begin{lstlisting}[language=sh]
pip install squarify
conda install -c conda-forge geopandas
conda install -c conda-forge wordcloud
\end{lstlisting}

\section{Análisis de eventos}

En esta sección se propone analizar los distintos tipos de eventos realizados por los usuarios de Trocafone. 

El campo \texttt{event} puede adquirir distintos tipos de valores categóricos que se describen como sigue:

\begin{itemize}
	\item \textbf{"viewed product"}: El usuario visita una página de producto.
	\item \textbf{"brand listing"}: El usuario visita un listado específico de una marca viendo un conjunto de productos.
	\item \textbf{"visited site"}:  El usuario ingresa al sitio a una determinada url.
	\item \textbf{"ad campaign hit"}: El usuario ingresa al sitio mediante una campana de marketing online.
	\item \textbf{"generic listing"}: El usuario visita la homepage.
	\item \textbf{"searched products"}: El usuario realiza una búsqueda de productos en la interfaz de búsqueda del site.
	\item \textbf{"search engine hit"}: El usuario ingresa al sitio mediante un motor de búsqueda web.
	\item \textbf{"checkout"}: El usuario ingresa al checkout de compra de un producto.
	\item \textbf{"static page"}:  El usuario visita una página.
	\item \textbf{"conversion"}: El usuario realiza una conversión, comprando un producto.
	\item \textbf{"lead"}: El usuario se registra para recibir una notificación de disponibilidad de stock, para un producto que no se encontraba disponible en ese momento.					
\end{itemize}

\subsection{Hipótesis sobre el truncamiento de los datos}

Se sabe por información de la consigna que el dataset proporcionado no representa el conjunto total de datos de todos los eventos realizados por los usuarios en el período de tiempo determinado. Es por esta razón que se busca elaborar una hipótesis en base al criterio con el que se fijó la selección de los datos. A partir de un análisis de los mismos se observó que todos los usuarios registrados en el dataset realizaron al menos 1 checkout, por lo que la base de datos original se truncó. Sin ir más lejos es evidente que no todos los usuarios que ingresan al sitio de Trocafone van a realizar un checkout.

De esta manera es importante remarcar que las conclusiones a las que se llegará en el desarrollo del Trabajo Práctico se basan en un sector segmentado de los datos, por lo que en ciertos aspectos del análisis no se podrá arribar a conclusiones fundadas.

\subsection{Frecuencia de eventos}

Se analiza qué tipo de evento es el más frecuente en el dataset. Para ello se grafica la cantidad registrada de eventos en función de los distintos tipos de eventos. Se excluyen del análisis los eventos \texttt{checkout} y \texttt{conversion} ya que serán analizados en detalle a lo largo del informe y los eventos \texttt{lead} y \texttt{static page} ya que no se consideran relevantes al análisis que se quiere realizar.

\begin{figure}[h!]
	\includegraphics[width=\textwidth]{figures/02-eventos-barplot.png}
	\caption{Frecuencia de eventos}
	\label{fig:freqeventos}
\end{figure}

Se observa en el gráfico que la mayor cantidad de eventos se relacionan a la vista de un producto, lo cual era previsible ya que Trocafone es una plataforma de e-commerce y ver productos constituye su principal función como sitio.

\subsection{Tráfico del sitio de acuerdo al mes y al día}

Otro aspecto a analizar es la cantidad de eventos producidos en cada día de la semana y del mes. Se busca detectar si se mantiene algún comportamiento específico a lo largo de los meses o si la cantidad de eventos registrada depende de algún factor temporal.

Para realizar este gráfico los datos fueron normalizados para evitar llegar a la conclusión que el mes con una mayor cantidad de eventos es el mes con más eventos por día. 

Se desprende del gráfico que la cantidad de eventos registrada no presenta ningún comportamiento específico. Se observa que dicha cantidad aumenta en la segunda quincena de cada mes pero se considera que la diferencia con el resto de los días no tiene la magnitud suficiente como para extraer alguna conclusión fundada. 

\begin{figure}[h!]
	\includegraphics[width=\linewidth]{figures/030-eventos_segun_mes-heatmap.png}
	\caption{Eventos segun mes y día}
	\label{fig:mesdiasnormalizado}
\end{figure}

\subsection{Tráfico del sitio de acuerdo al mes y al día de la semana}

En este apartado se busca analizar si algún día de la semana se registra una mayor cantidad de eventos. 

\begin{figure}[h!]
	\includegraphics[width=\linewidth]{figures/031-eventos_segun_dow-heatmap.png}
	\caption{Eventos segun mes y día}
	\label{fig:messemanasnormalizado}
\end{figure}

Se realiza un gráfico del mismo estilo que el anterior y se normaliza por las mismas razones.

Es notable que durante los días hábiles de la semana el tráfico es mucho mayor que al fin de semana. Esto puede deberse a que los fines de semana suelen ser días de descanso, donde la gente puede no estar pensando en realizar una compra, además de no poder retirarla. En la semana aumenta el tráfico debido a que el envío o el retiro del celular puede realizarse en el momento.

\subsection{Tráfico del sitio según mes}

En este apartado se busca analizar si en algún mes se registró una mayor cantidad de eventos o si la distribución de las visitas fue uniforme a lo largo del tiempo.

\begin{figure}[h!]
	\includegraphics[width=\linewidth]{figures/032-eventos_segun_mes-barplot.png}
	\caption{Eventos segun mes}
	\label{fig:mes}
\end{figure}

Los meses de mayo y junio registraron una cantidad notablemente mayor de eventos. Este resultado llama la antención por lo que se analiza en mayor profundidad en la próxima sección.

\subsubsection{¿Por qué mayo y junio registran una mayor cantidad de eventos?}

\newpage
\appendix


\newpage

\end{document}
