\documentclass[a4paper]{article}

\usepackage[utf8]{inputenc}
\usepackage[spanish]{babel}
\usepackage{float}
\usepackage{fancyhdr,graphicx,listings,amsmath,tocloft}
\usepackage[colorlinks=true,linkcolor=black,urlcolor=blue,bookmarksopen=true]{hyperref}

\newcommand{\materia}{[75.06 / 95.58] Organización de Datos}
\newcommand{\trabajo}{Trabajo Práctico 2: Machine Learning}
\newcommand{\trabajoheader}{TP2}
\newcommand{\cuatri}{2c2018}
\newcommand{\cuatrimestre}{Segundo cuatrimestre de 2018}
\newcommand{\grupo}{Grupo Datatouille}
\newcommand{\repo}{https://github.com/FdelMazo/7506-Datos/}
\newcommand{\kernel}{https://kaggle.com/datatouille2018/}
\newcommand{\alumnos}{
    Bojman, Camila & 101055 &  camiboj@gmail.com\\
    del Mazo, Federico & 100029 & delmazofederico@gmail.com\\
    Hortas, Cecilia & 100687 & ceci.hortas@gmail.com\\
    Souto, Rodrigo & 97649 & rnsoutob@gmail.com\\
}
\newcommand{\curso}{Curso 01}
\newcommand{\docentes}{
    \item Argerich, Luis Argerich
    \item Golmar, Natalia
    \item Martinelli, Damina Ariel
    \item Ramos Mejia, Martín Gabriel
}

\hypersetup{
    pdftitle={\trabajo},
	pdfsubject={\materia},
	pdfauthor={\grupo},
}

\setlength{\cftbeforesecskip}{6pt}

\pagestyle{fancy}
\fancyhf{}
\fancyhead[L]{\materia}
\fancyhead[R]{\trabajoheader - \cuatri}
\renewcommand{\headrulewidth}{0.4pt}
\fancyfoot[C]{\thepage}
\renewcommand{\footrulewidth}{0.4pt}

\begin{document}
\pagenumbering{gobble}

\begin{titlepage}
	\hfill\includegraphics[width=6cm]{fiuba.jpg}
    \begin{center}
    \vfill
    \Huge \textbf{\trabajo}
    \vskip2cm
    \Large \materia\\
    \cuatrimestre
    \vfill
    \begin{flushleft} 
    \grupo
    \end{flushleft}
    \begin{tabular}{|l|c|r|}
	\hline
	Alumno & Padrón & Mail\\
	\hline \hline
    \alumnos
	\hline
	\end{tabular}
    \begin{flushleft} 
    \large{\url{\repo}} \\
    \large{\url{\kernel}} \\
    \end{flushleft}
    \vskip1cm
    \end{center}
    \curso
    \begin{itemize}
        \docentes
    \end{itemize}
\end{titlepage}
\pagenumbering{roman}
\tableofcontents
\newpage
\pagenumbering{arabic}
\setcounter{page}{1}

\section{Introducción}

Se propone exponer en el siguiente informe el desarrollo del Trabajo Práctico para predecir la probabilidad de que un usuario de Trocafone realice una conversión en un período determinado de tiempo. Con dicho propósito se utilizan como base dos archivos csv brindados por la empresa. En un primer lugar el archivo \texttt{events\_up\_to\_01062018.csv} que contiene la información de los eventos realizados por un conjunto de usuarios hasta el 31 de mayo de 2018. En un segundo lugar el archivo \texttt{labels\_training\_set.csv} que determina si un conjunto de usuarios realizó o no una conversión desde el 01/06/2018 hasta el 15/06/2018. 

Se propone como objetivo desarrollar una métrica a fin de evaluar los distintos resultados obtenidos. Se presentan en el informe las distintas decisiones adoptadas para elegir un resultado por sobre otro. Así mismo, se propuso utilizar distintos algoritmos de la librería sklearn para elegir el que modelice mejor. 

Por otro lado se crearon distintos csv a fin de poder desarrollar de mejor manera el proceso de feature engineering y obtener un pre-procesamiento de los datos que permita encontrar los resultados más altos. 

\section{Investigación}

Como se mencionó en la \textit{Introducción}, se realizó un proceso de exploración de los datos para crear nuevos atributos y extraer nuevos features. De esta manera se crearon distintos csv con información extraída tanto de internet como de los propios datos para luego concatenar al set de datos final. Se procede a dar una breve explicación de la funcionalidad de cada uno de ellos. Muchos de ellos se encuentran detallados en el 'Notebook Anexo' del TP1\footnote{https://fdelmazo.github.io/7506-Datos/TP1/TP1.html}.

\begin{itemize}
	\item brands.csv

Se agregó una columna al dataframe que detalla qué marca está involucarada en el evento del usuario.

	\item {os.csv}
	
Se agregó una columna al dataframe que detalla qué sistema operativo está involucarada en el evento del usuario.

	\item{browsers.csv}

Se agregó una columna al dataframe que detalla qué explorador de internet se accede al sitio.	

	\item{sessions.csv}

Se agregó el concepto de sesión, que se define como la agrupación de una serie de eventos por usuario, los cuales están todos con menos de 30 minutos de
inactividad entre el actual y el anterior. Esto fue fijado con un criterio arbitrario a fin de poder discretizar el tiempo y definir este concepto.

	\item{prices.csv}

Se agregó una columna al dataframe que indica el precio del producto involucrado en el evento del usuario. Para ello se extrajeron los precios de la página de Trocafone\footnote{https://www.trocafone.com/} considerando el sku, el modelo, el color, la capacidad de almacenamiento y la condición.
\end{itemize}

\section{Feature engineering}

A partir del nuevo dataframe obtenido con la unión de todos los csv descritos en el inciso anterior se procedió a la búsqueda de features. En esta étapa del desarrollo del Trabajo Práctico se buscó explotar las distintas ideas y después con un proceso de selección que será explicado maś adelante elegir los features pertinentes y más útiles al modelo. 

\subsection{Features básicos}
Se detallan los features generales considerados como pertinentes al modelo.

\begin{itemize}
	\item is\_viewed\_product: el usuario vió un producto
	\item is\_checkout: el usuario llegó a checkout con un producto
	\item is\_conversion: el usuario compró un producto
	\item session\_checkout\_first: el usuario en su primera sesión realizó un checkout
	\item session\_conversion\_first: el usuario en su primera sesión realizó una conversión
	\item session\_ad\_first: el usuario en su primera sesión llegó con una campaña publicitaria
	\item session\_ad\_checkout\_event: el usuario en su primera sesión llegó con una campaña publicitaria
	
\end{itemize}








\section{Algoritmos utilizados}

\section{Desarrollo}

\section{Resultados obtenidos}

\section{Conclusiones}

\end{document}